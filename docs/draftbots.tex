\documentclass[oneside,12pt]{amsart}
\usepackage[T1]{fontenc}
\usepackage[latin9]{inputenc}
\usepackage{amssymb}
\usepackage[natbibapa]{apacite}
\usepackage{enumerate}
\usepackage{hyperref}
\usepackage{listings}
\usepackage{placeins}
\usepackage{url}

\newtheorem*{definition}{Definition}
\newtheorem*{remark}{Remark}
\newtheorem*{note}{Note}
\newtheorem{theorem}{Theorem}[section]
\newtheorem{lemma}{Lemma}[section]
\newtheorem{corollary}{Corollary}[section]
\newtheorem{conjecture}{Conjecture}[section]

\DeclareMathOperator{\elo}{Elo}

\title{Optimality of Draft Bot Strategies}
\author{ruler501}
\date{\today}

%\everymath{\displaystyle}
\begin{document}

\begin{abstract}
    We seek to formalize the notion of an optimal draft starting from the
    notion that it should ensure all players have pools which can produce
    roughly equal power level decks. We then ecpand on the notion of
    optimality to introduce algorithms for the bots to follow to macimize
    the optimality of the resultant pools from the draft.
\end{abstract}

\maketitle

\tableofcontents

\section{Introduction}

    \subsection{Current Algorithms}
        The first step to assessing power level is to create a concept of card power
        level. CubeCobra currently has a system for this based on the ELO rating
        system traditionally used for chess \citenum{WikiELO}. In this system we
        assign each card $c_i \in Cards$ a time-parametrized rating
        $elo_t: Cards \to \mathbb R$  for the which the following holds.
        \begin{equation}\label{eqn:ZeroSumElo}
            \forall t \in \mathbb R, \sum_{c_i \in Cards} \elo_t(c_i) = C
        \end{equation}
        From this value we define the quality of a card, $\operatorname{Q}_t: Cards \to \mathbb R^+$ to be:
        \begin{equation}\label{eqn:LogisticEloQuality}
            \operatorname{Q}_t(c_i) = 10^{\frac{\elo_t(c_i)}{400}}
        \end{equation}
        We define the predicted probability of $c_i$ being picked over $c_j$ to be
        assess picks as pairwise competitions where picking $c_i$ from $\{c_1,...,c_n\}$
        is seen as $c_i$ winning a head to head competition against each of the other
        members of the pack.

\section{A Formal Definition of Optimality}

\section{Proposal for Better Color Assignment}

\section{Proposal for Live Adjustment of Color Assignments}

\bibliographystyle{apacite}
\bibliography{draftbots}
\end{document}
